%%%%%%%%%%%%%%%%%%%%%%%%%%%%%%%%%%%%%%%%%%%%%%%%%%%%%%%%%%%%%%%%%%%%%%%%%%%%%%%%
%%%%%%%%%%%%%%%%%%%%%%%%%%%% BSUITE REPORT TEMPLATE %%%%%%%%%%%%%%%%%%%%%%%%%%%%
%%%%%%%%%%%%%%%%%%%%%%%%%%%%%%%%%%%%%%%%%%%%%%%%%%%%%%%%%%%%%%%%%%%%%%%%%%%%%%%%
%
% Use this LaTeX code to generate an automatic bsuite appendix for your paper
% submission. First %%%%%%%%%%%%%%%%%%%%%%%%%%%%%%%%%%%%%%%%%%%%%%%%%%%%%%%%%%%%%%%%%%%%%%%%%%%%%%%%
%%%%%%%%%%%%%%%%%%%%%%%%%%%% BSUITE LATEX PREAMBLE %%%%%%%%%%%%%%%%%%%%%%%%%%%%%
%%%%%%%%%%%%%%%%%%%%%%%%%%%%%%%%%%%%%%%%%%%%%%%%%%%%%%%%%%%%%%%%%%%%%%%%%%%%%%%%
%
% Use this LaTeX code to generate an automatic bsuite appendix for your paper
% submission. First%%%%%%%%%%%%%%%%%%%%%%%%%%%%%%%%%%%%%%%%%%%%%%%%%%%%%%%%%%%%%%%%%%%%%%%%%%%%%%%%
%%%%%%%%%%%%%%%%%%%%%%%%%%%% BSUITE LATEX PREAMBLE %%%%%%%%%%%%%%%%%%%%%%%%%%%%%
%%%%%%%%%%%%%%%%%%%%%%%%%%%%%%%%%%%%%%%%%%%%%%%%%%%%%%%%%%%%%%%%%%%%%%%%%%%%%%%%
%
% Use this LaTeX code to generate an automatic bsuite appendix for your paper
% submission. First%%%%%%%%%%%%%%%%%%%%%%%%%%%%%%%%%%%%%%%%%%%%%%%%%%%%%%%%%%%%%%%%%%%%%%%%%%%%%%%%
%%%%%%%%%%%%%%%%%%%%%%%%%%%% BSUITE LATEX PREAMBLE %%%%%%%%%%%%%%%%%%%%%%%%%%%%%
%%%%%%%%%%%%%%%%%%%%%%%%%%%%%%%%%%%%%%%%%%%%%%%%%%%%%%%%%%%%%%%%%%%%%%%%%%%%%%%%
%
% Use this LaTeX code to generate an automatic bsuite appendix for your paper
% submission. First\input{bsuite_preamble} before your \begin{document}, you can
% fill in the custom \buitecolab, \bsuiteradarplot, \bsuitebarplot to link to
% the necessary bsuite assets for publication.
%
% Next, fill in the necessary sections of bsuite_appendix, and either copy/paste
% or \input{} this into your conference file to create an appendix page.
%
% For some conference formats (e.g. ICLR) it is important to allow margin change
% \includepackage{geometry}, we do not include this in the preamble by default.
%
% Remember that \input{bsuite_preamble.tex} is essentially equivalent to
% copy/paste... and in some cases that approach will be much easier to debug!

\usepackage{caption}
\usepackage{changepage}
\usepackage{enumitem}
\usepackage{graphicx}


%%%%%%%%%%%%%%%%%%%%%%%%%%%%%%%%%%%%%%%%%%%%%%%%%%%%%%%%%%%%%%%%%%%%%%%%%%%%%%%%
% DO NOT CHANGE THESE LINKS
%
% These are useful commands that are used in the bsuite_appendix.
% You should not change these from their default values.

\newcommand{\bsuite}{\texttt{bsuite}}
\newcommand{\bsuiteversion}{\texttt{bsuite2019}}
\newcommand{\bsuitegithub}{\url{github.com/deepmind/bsuite}}

% This is a Macro that returns a title = bsuite report: Your Paper Title.
\newcommand{\bsuitetitle}[1]{
\rule{\linewidth}{4pt}
\vspace{-5mm}

\section{\hfil \LARGE \normalfont
\bsuite\ report: #1
\vspace{2mm} \hfil
\vspace{-3mm}
}

\rule{\linewidth}{1pt}
}

\newcommand{\bsuiteabstract}{
{\small
\begin{adjustwidth}{1.5cm}{1.5cm}
The \textit{Behaviour Suite for Core Reinforcement Learning}, or \bsuite\ for short, is a collection of carefully-designed experiments that investigate core capabilities of a reinforcement learning (RL) agent.
The aim of the \bsuite\ project is to collect clear, informative and scalable problems that capture key issues in the design of efficient and general learning algorithms and study agent behaviour through their performance on these shared benchmarks.
This report provides a snapshot of agent performance on \bsuiteversion, obtained by running the experiments from \bsuitegithub\ \cite{osband2019bsuite}.
\end{adjustwidth}
}
}

%%%%%%%%%%%%%%%%%%%%%%%%%%%%%%%%%%%%%%%%%%%%%%%%%%%%%%%%%%%%%%%%%%%%%%%%%%%%%%%%
% CHANGE THESE LINKS
%
% These are convenience macros that provide links to you bsuite material.
% Remember that paths to images should be given *relative* to the file that
% is \input{bsuite_appendix}. In some cases this will be easier to debug
% if you just copy/paste the tex into your paper file.

\newcommand{\bsuitecolab}{\href{YOUR-LINK-HERE}}  % full bsuite colab report
\newcommand{\bsuiteradarplot}{images/radar_plot}  % path to radar plot
\newcommand{\bsuitebarplot}{images/bar_plot}  % path to bar plot


 before your \begin{document}, you can
% fill in the custom \buitecolab, \bsuiteradarplot, \bsuitebarplot to link to
% the necessary bsuite assets for publication.
%
% Next, fill in the necessary sections of bsuite_appendix, and either copy/paste
% or \input{} this into your conference file to create an appendix page.
%
% For some conference formats (e.g. ICLR) it is important to allow margin change
% \includepackage{geometry}, we do not include this in the preamble by default.
%
% Remember that %%%%%%%%%%%%%%%%%%%%%%%%%%%%%%%%%%%%%%%%%%%%%%%%%%%%%%%%%%%%%%%%%%%%%%%%%%%%%%%%
%%%%%%%%%%%%%%%%%%%%%%%%%%%% BSUITE LATEX PREAMBLE %%%%%%%%%%%%%%%%%%%%%%%%%%%%%
%%%%%%%%%%%%%%%%%%%%%%%%%%%%%%%%%%%%%%%%%%%%%%%%%%%%%%%%%%%%%%%%%%%%%%%%%%%%%%%%
%
% Use this LaTeX code to generate an automatic bsuite appendix for your paper
% submission. First\input{bsuite_preamble} before your \begin{document}, you can
% fill in the custom \buitecolab, \bsuiteradarplot, \bsuitebarplot to link to
% the necessary bsuite assets for publication.
%
% Next, fill in the necessary sections of bsuite_appendix, and either copy/paste
% or \input{} this into your conference file to create an appendix page.
%
% For some conference formats (e.g. ICLR) it is important to allow margin change
% \includepackage{geometry}, we do not include this in the preamble by default.
%
% Remember that \input{bsuite_preamble.tex} is essentially equivalent to
% copy/paste... and in some cases that approach will be much easier to debug!

\usepackage{caption}
\usepackage{changepage}
\usepackage{enumitem}
\usepackage{graphicx}


%%%%%%%%%%%%%%%%%%%%%%%%%%%%%%%%%%%%%%%%%%%%%%%%%%%%%%%%%%%%%%%%%%%%%%%%%%%%%%%%
% DO NOT CHANGE THESE LINKS
%
% These are useful commands that are used in the bsuite_appendix.
% You should not change these from their default values.

\newcommand{\bsuite}{\texttt{bsuite}}
\newcommand{\bsuiteversion}{\texttt{bsuite2019}}
\newcommand{\bsuitegithub}{\url{github.com/deepmind/bsuite}}

% This is a Macro that returns a title = bsuite report: Your Paper Title.
\newcommand{\bsuitetitle}[1]{
\rule{\linewidth}{4pt}
\vspace{-5mm}

\section{\hfil \LARGE \normalfont
\bsuite\ report: #1
\vspace{2mm} \hfil
\vspace{-3mm}
}

\rule{\linewidth}{1pt}
}

\newcommand{\bsuiteabstract}{
{\small
\begin{adjustwidth}{1.5cm}{1.5cm}
The \textit{Behaviour Suite for Core Reinforcement Learning}, or \bsuite\ for short, is a collection of carefully-designed experiments that investigate core capabilities of a reinforcement learning (RL) agent.
The aim of the \bsuite\ project is to collect clear, informative and scalable problems that capture key issues in the design of efficient and general learning algorithms and study agent behaviour through their performance on these shared benchmarks.
This report provides a snapshot of agent performance on \bsuiteversion, obtained by running the experiments from \bsuitegithub\ \cite{osband2019bsuite}.
\end{adjustwidth}
}
}

%%%%%%%%%%%%%%%%%%%%%%%%%%%%%%%%%%%%%%%%%%%%%%%%%%%%%%%%%%%%%%%%%%%%%%%%%%%%%%%%
% CHANGE THESE LINKS
%
% These are convenience macros that provide links to you bsuite material.
% Remember that paths to images should be given *relative* to the file that
% is \input{bsuite_appendix}. In some cases this will be easier to debug
% if you just copy/paste the tex into your paper file.

\newcommand{\bsuitecolab}{\href{YOUR-LINK-HERE}}  % full bsuite colab report
\newcommand{\bsuiteradarplot}{images/radar_plot}  % path to radar plot
\newcommand{\bsuitebarplot}{images/bar_plot}  % path to bar plot


 is essentially equivalent to
% copy/paste... and in some cases that approach will be much easier to debug!

\usepackage{caption}
\usepackage{changepage}
\usepackage{enumitem}
\usepackage{graphicx}


%%%%%%%%%%%%%%%%%%%%%%%%%%%%%%%%%%%%%%%%%%%%%%%%%%%%%%%%%%%%%%%%%%%%%%%%%%%%%%%%
% DO NOT CHANGE THESE LINKS
%
% These are useful commands that are used in the bsuite_appendix.
% You should not change these from their default values.

\newcommand{\bsuite}{\texttt{bsuite}}
\newcommand{\bsuiteversion}{\texttt{bsuite2019}}
\newcommand{\bsuitegithub}{\url{github.com/deepmind/bsuite}}

% This is a Macro that returns a title = bsuite report: Your Paper Title.
\newcommand{\bsuitetitle}[1]{
\rule{\linewidth}{4pt}
\vspace{-5mm}

\section{\hfil \LARGE \normalfont
\bsuite\ report: #1
\vspace{2mm} \hfil
\vspace{-3mm}
}

\rule{\linewidth}{1pt}
}

\newcommand{\bsuiteabstract}{
{\small
\begin{adjustwidth}{1.5cm}{1.5cm}
The \textit{Behaviour Suite for Core Reinforcement Learning}, or \bsuite\ for short, is a collection of carefully-designed experiments that investigate core capabilities of a reinforcement learning (RL) agent.
The aim of the \bsuite\ project is to collect clear, informative and scalable problems that capture key issues in the design of efficient and general learning algorithms and study agent behaviour through their performance on these shared benchmarks.
This report provides a snapshot of agent performance on \bsuiteversion, obtained by running the experiments from \bsuitegithub\ \cite{osband2019bsuite}.
\end{adjustwidth}
}
}

%%%%%%%%%%%%%%%%%%%%%%%%%%%%%%%%%%%%%%%%%%%%%%%%%%%%%%%%%%%%%%%%%%%%%%%%%%%%%%%%
% CHANGE THESE LINKS
%
% These are convenience macros that provide links to you bsuite material.
% Remember that paths to images should be given *relative* to the file that
% is %%%%%%%%%%%%%%%%%%%%%%%%%%%%%%%%%%%%%%%%%%%%%%%%%%%%%%%%%%%%%%%%%%%%%%%%%%%%%%%%
%%%%%%%%%%%%%%%%%%%%%%%%%%%% BSUITE REPORT TEMPLATE %%%%%%%%%%%%%%%%%%%%%%%%%%%%
%%%%%%%%%%%%%%%%%%%%%%%%%%%%%%%%%%%%%%%%%%%%%%%%%%%%%%%%%%%%%%%%%%%%%%%%%%%%%%%%
%
% Use this LaTeX code to generate an automatic bsuite appendix for your paper
% submission. First \input{bsuite_preamble} before your \begin{document}.
% You can use the bsuite_preamble to define the location of your plots, plus a
% link to the full bsuite comments.
%
% Then, write a short description of your agents in app:bsuite-agents, and a short
% commentary on your results in app:bsuite-commentary.
%
% For some conference formats (e.g. ICLR) it is important to allow margin change
% \includepackage{geometry}, we do not include this in the preamble by default.
%


\newpage
\onecolumn

% If the package does not allow geometry, do not fail
\ifx\newgeometry\undefined\else
\newgeometry{top=20mm, bottom=20mm, left=20mm, right=20mm}
\fi


%%%%%%%%%%%%%%%%%%%%%%%%%%%%%%%%%%%%%%%%%%%%%%%%%%%%%%%%%%%%%%%%%%%%%%%%%%%%%%%%
% TITLE + ABSTRACT [ADD PAPER TITLE]
%
% Macros are defined in bsuite_preamble.tex, use the \label{} to \ref{} from
% other sections in your paper.
\bsuitetitle{Your Paper Title}
\label{app:bsuite-report}
\bsuiteabstract


%%%%%%%%%%%%%%%%%%%%%%%%%%%%%%%%%%%%%%%%%%%%%%%%%%%%%%%%%%%%%%%%%%%%%%%%%%%%%%%%
% AGENT DEFINITION [EDIT]
%
% Use this section to provide a brief overview of the agents that you run on
% bsuite. Usually this will involve links to full descriptions elsewhere in
% your paper.

\subsection{Agent definition}
\label{app:bsuite-agents}
In this experiment all implementations are taken from \url{github.com/deepmind/bsuite/baselines} with default configurations.
We provide a brief summary of the agents run on \bsuiteversion:
\begin{itemize}[noitemsep, nolistsep]
    \item {\bf random}: selects action uniformly at random each timestep.
    \item {\bf dqn}: Deep Q-networks \cite{mnih2015human}.
    \item {\bf boot\_dqn}: bootstrapped DQN with prior networks \cite{osband2016deep,osband2018rpf}.
    \item {\bf actor\_critic\_rnn}: an actor critic with recurrent neural network \cite{mnih2016asynchronous}.
\end{itemize}



%%%%%%%%%%%%%%%%%%%%%%%%%%%%%%%%%%%%%%%%%%%%%%%%%%%%%%%%%%%%%%%%%%%%%%%%%%%%%%%%
% SUMMARY SCORES [DO NOT EDIT]
\subsection{Summary scores}
\label{app:bsuite-scores}

Each \bsuite\ experiment outputs a summary score in [0,1].
We aggregate these scores by according to key experiment type, according to the standard analysis notebook.
A detailed analysis of each of these experiments may be found in a notebook hosted on Colaboratory: \bsuitecolab.

\ifx\newgeometry\undefined\vspace{-2mm}\else\fi % Squeezing for ICML

\begin{figure}[h!]
\centering
\begin{minipage}[t]{.5\textwidth}
  \centering
  \includegraphics[width=\textwidth,height=60mm,keepaspectratio]{\bsuiteradarplot}
  \captionof{figure}{Radar plot gives a snapshot of agent behaviour.}
  \label{fig:radar}
\end{minipage}%
\begin{minipage}[t]{.5\textwidth}
  \centering
  \includegraphics[width=\textwidth,height=60mm,keepaspectratio]{\bsuitebarplot}
  \captionof{figure}{Summary score for each \bsuite\ experiment.}
  \label{fig:bar}
\end{minipage}
\end{figure}

\ifx\newgeometry\undefined\vspace{-2mm}\else\fi % Squeezing for ICML

%%%%%%%%%%%%%%%%%%%%%%%%%%%%%%%%%%%%%%%%%%%%%%%%%%%%%%%%%%%%%%%%%%%%%%%%%%%%%%%%
% RESULTS COMMENTARY [EDIT]

\subsection{Results commentary}
\label{app:bsuite-commentary}

\begin{itemize}[noitemsep, nolistsep, leftmargin=*]
    \item {\bf random} performs poorly across all aspects.
    This confirms that our scoring functions are working as intended.
    \item {\bf dqn} performs well on basic tasks, and quite well on credit assignment, generalization, noise and scale.
    DQN performs extremely poorly across memory and exploration tasks.
    The feedforward MLP has no mechanism for memory, and $\epsilon$=5\%-greedy action selection is notoriously inefficient in domains that require efficient exploration.
    \item {\bf boot\_dqn} performs mostly identically to DQN, except for exploration where it greatly outperforms, and also a smaller boost to performance under noise.
    This result matches our understanding of Bootstrapped DQN as a variant of DQN designed to estimate uncertainty and use this to guide deep exploration.
    \item {\bf actor\_critic\_rnn} typically performs worse than either DQN or Bootstrapped DQN on all tasks apart from memory.
    This agent is the only one able to perform better than random due to its recurrent network architecture.
\end{itemize}


\newpage
. In some cases this will be easier to debug
% if you just copy/paste the tex into your paper file.

\newcommand{\bsuitecolab}{\href{YOUR-LINK-HERE}}  % full bsuite colab report
\newcommand{\bsuiteradarplot}{images/radar_plot}  % path to radar plot
\newcommand{\bsuitebarplot}{images/bar_plot}  % path to bar plot


 before your \begin{document}, you can
% fill in the custom \buitecolab, \bsuiteradarplot, \bsuitebarplot to link to
% the necessary bsuite assets for publication.
%
% Next, fill in the necessary sections of bsuite_appendix, and either copy/paste
% or \input{} this into your conference file to create an appendix page.
%
% For some conference formats (e.g. ICLR) it is important to allow margin change
% \includepackage{geometry}, we do not include this in the preamble by default.
%
% Remember that %%%%%%%%%%%%%%%%%%%%%%%%%%%%%%%%%%%%%%%%%%%%%%%%%%%%%%%%%%%%%%%%%%%%%%%%%%%%%%%%
%%%%%%%%%%%%%%%%%%%%%%%%%%%% BSUITE LATEX PREAMBLE %%%%%%%%%%%%%%%%%%%%%%%%%%%%%
%%%%%%%%%%%%%%%%%%%%%%%%%%%%%%%%%%%%%%%%%%%%%%%%%%%%%%%%%%%%%%%%%%%%%%%%%%%%%%%%
%
% Use this LaTeX code to generate an automatic bsuite appendix for your paper
% submission. First%%%%%%%%%%%%%%%%%%%%%%%%%%%%%%%%%%%%%%%%%%%%%%%%%%%%%%%%%%%%%%%%%%%%%%%%%%%%%%%%
%%%%%%%%%%%%%%%%%%%%%%%%%%%% BSUITE LATEX PREAMBLE %%%%%%%%%%%%%%%%%%%%%%%%%%%%%
%%%%%%%%%%%%%%%%%%%%%%%%%%%%%%%%%%%%%%%%%%%%%%%%%%%%%%%%%%%%%%%%%%%%%%%%%%%%%%%%
%
% Use this LaTeX code to generate an automatic bsuite appendix for your paper
% submission. First\input{bsuite_preamble} before your \begin{document}, you can
% fill in the custom \buitecolab, \bsuiteradarplot, \bsuitebarplot to link to
% the necessary bsuite assets for publication.
%
% Next, fill in the necessary sections of bsuite_appendix, and either copy/paste
% or \input{} this into your conference file to create an appendix page.
%
% For some conference formats (e.g. ICLR) it is important to allow margin change
% \includepackage{geometry}, we do not include this in the preamble by default.
%
% Remember that \input{bsuite_preamble.tex} is essentially equivalent to
% copy/paste... and in some cases that approach will be much easier to debug!

\usepackage{caption}
\usepackage{changepage}
\usepackage{enumitem}
\usepackage{graphicx}


%%%%%%%%%%%%%%%%%%%%%%%%%%%%%%%%%%%%%%%%%%%%%%%%%%%%%%%%%%%%%%%%%%%%%%%%%%%%%%%%
% DO NOT CHANGE THESE LINKS
%
% These are useful commands that are used in the bsuite_appendix.
% You should not change these from their default values.

\newcommand{\bsuite}{\texttt{bsuite}}
\newcommand{\bsuiteversion}{\texttt{bsuite2019}}
\newcommand{\bsuitegithub}{\url{github.com/deepmind/bsuite}}

% This is a Macro that returns a title = bsuite report: Your Paper Title.
\newcommand{\bsuitetitle}[1]{
\rule{\linewidth}{4pt}
\vspace{-5mm}

\section{\hfil \LARGE \normalfont
\bsuite\ report: #1
\vspace{2mm} \hfil
\vspace{-3mm}
}

\rule{\linewidth}{1pt}
}

\newcommand{\bsuiteabstract}{
{\small
\begin{adjustwidth}{1.5cm}{1.5cm}
The \textit{Behaviour Suite for Core Reinforcement Learning}, or \bsuite\ for short, is a collection of carefully-designed experiments that investigate core capabilities of a reinforcement learning (RL) agent.
The aim of the \bsuite\ project is to collect clear, informative and scalable problems that capture key issues in the design of efficient and general learning algorithms and study agent behaviour through their performance on these shared benchmarks.
This report provides a snapshot of agent performance on \bsuiteversion, obtained by running the experiments from \bsuitegithub\ \cite{osband2019bsuite}.
\end{adjustwidth}
}
}

%%%%%%%%%%%%%%%%%%%%%%%%%%%%%%%%%%%%%%%%%%%%%%%%%%%%%%%%%%%%%%%%%%%%%%%%%%%%%%%%
% CHANGE THESE LINKS
%
% These are convenience macros that provide links to you bsuite material.
% Remember that paths to images should be given *relative* to the file that
% is \input{bsuite_appendix}. In some cases this will be easier to debug
% if you just copy/paste the tex into your paper file.

\newcommand{\bsuitecolab}{\href{YOUR-LINK-HERE}}  % full bsuite colab report
\newcommand{\bsuiteradarplot}{images/radar_plot}  % path to radar plot
\newcommand{\bsuitebarplot}{images/bar_plot}  % path to bar plot


 before your \begin{document}, you can
% fill in the custom \buitecolab, \bsuiteradarplot, \bsuitebarplot to link to
% the necessary bsuite assets for publication.
%
% Next, fill in the necessary sections of bsuite_appendix, and either copy/paste
% or \input{} this into your conference file to create an appendix page.
%
% For some conference formats (e.g. ICLR) it is important to allow margin change
% \includepackage{geometry}, we do not include this in the preamble by default.
%
% Remember that %%%%%%%%%%%%%%%%%%%%%%%%%%%%%%%%%%%%%%%%%%%%%%%%%%%%%%%%%%%%%%%%%%%%%%%%%%%%%%%%
%%%%%%%%%%%%%%%%%%%%%%%%%%%% BSUITE LATEX PREAMBLE %%%%%%%%%%%%%%%%%%%%%%%%%%%%%
%%%%%%%%%%%%%%%%%%%%%%%%%%%%%%%%%%%%%%%%%%%%%%%%%%%%%%%%%%%%%%%%%%%%%%%%%%%%%%%%
%
% Use this LaTeX code to generate an automatic bsuite appendix for your paper
% submission. First\input{bsuite_preamble} before your \begin{document}, you can
% fill in the custom \buitecolab, \bsuiteradarplot, \bsuitebarplot to link to
% the necessary bsuite assets for publication.
%
% Next, fill in the necessary sections of bsuite_appendix, and either copy/paste
% or \input{} this into your conference file to create an appendix page.
%
% For some conference formats (e.g. ICLR) it is important to allow margin change
% \includepackage{geometry}, we do not include this in the preamble by default.
%
% Remember that \input{bsuite_preamble.tex} is essentially equivalent to
% copy/paste... and in some cases that approach will be much easier to debug!

\usepackage{caption}
\usepackage{changepage}
\usepackage{enumitem}
\usepackage{graphicx}


%%%%%%%%%%%%%%%%%%%%%%%%%%%%%%%%%%%%%%%%%%%%%%%%%%%%%%%%%%%%%%%%%%%%%%%%%%%%%%%%
% DO NOT CHANGE THESE LINKS
%
% These are useful commands that are used in the bsuite_appendix.
% You should not change these from their default values.

\newcommand{\bsuite}{\texttt{bsuite}}
\newcommand{\bsuiteversion}{\texttt{bsuite2019}}
\newcommand{\bsuitegithub}{\url{github.com/deepmind/bsuite}}

% This is a Macro that returns a title = bsuite report: Your Paper Title.
\newcommand{\bsuitetitle}[1]{
\rule{\linewidth}{4pt}
\vspace{-5mm}

\section{\hfil \LARGE \normalfont
\bsuite\ report: #1
\vspace{2mm} \hfil
\vspace{-3mm}
}

\rule{\linewidth}{1pt}
}

\newcommand{\bsuiteabstract}{
{\small
\begin{adjustwidth}{1.5cm}{1.5cm}
The \textit{Behaviour Suite for Core Reinforcement Learning}, or \bsuite\ for short, is a collection of carefully-designed experiments that investigate core capabilities of a reinforcement learning (RL) agent.
The aim of the \bsuite\ project is to collect clear, informative and scalable problems that capture key issues in the design of efficient and general learning algorithms and study agent behaviour through their performance on these shared benchmarks.
This report provides a snapshot of agent performance on \bsuiteversion, obtained by running the experiments from \bsuitegithub\ \cite{osband2019bsuite}.
\end{adjustwidth}
}
}

%%%%%%%%%%%%%%%%%%%%%%%%%%%%%%%%%%%%%%%%%%%%%%%%%%%%%%%%%%%%%%%%%%%%%%%%%%%%%%%%
% CHANGE THESE LINKS
%
% These are convenience macros that provide links to you bsuite material.
% Remember that paths to images should be given *relative* to the file that
% is \input{bsuite_appendix}. In some cases this will be easier to debug
% if you just copy/paste the tex into your paper file.

\newcommand{\bsuitecolab}{\href{YOUR-LINK-HERE}}  % full bsuite colab report
\newcommand{\bsuiteradarplot}{images/radar_plot}  % path to radar plot
\newcommand{\bsuitebarplot}{images/bar_plot}  % path to bar plot


 is essentially equivalent to
% copy/paste... and in some cases that approach will be much easier to debug!

\usepackage{caption}
\usepackage{changepage}
\usepackage{enumitem}
\usepackage{graphicx}


%%%%%%%%%%%%%%%%%%%%%%%%%%%%%%%%%%%%%%%%%%%%%%%%%%%%%%%%%%%%%%%%%%%%%%%%%%%%%%%%
% DO NOT CHANGE THESE LINKS
%
% These are useful commands that are used in the bsuite_appendix.
% You should not change these from their default values.

\newcommand{\bsuite}{\texttt{bsuite}}
\newcommand{\bsuiteversion}{\texttt{bsuite2019}}
\newcommand{\bsuitegithub}{\url{github.com/deepmind/bsuite}}

% This is a Macro that returns a title = bsuite report: Your Paper Title.
\newcommand{\bsuitetitle}[1]{
\rule{\linewidth}{4pt}
\vspace{-5mm}

\section{\hfil \LARGE \normalfont
\bsuite\ report: #1
\vspace{2mm} \hfil
\vspace{-3mm}
}

\rule{\linewidth}{1pt}
}

\newcommand{\bsuiteabstract}{
{\small
\begin{adjustwidth}{1.5cm}{1.5cm}
The \textit{Behaviour Suite for Core Reinforcement Learning}, or \bsuite\ for short, is a collection of carefully-designed experiments that investigate core capabilities of a reinforcement learning (RL) agent.
The aim of the \bsuite\ project is to collect clear, informative and scalable problems that capture key issues in the design of efficient and general learning algorithms and study agent behaviour through their performance on these shared benchmarks.
This report provides a snapshot of agent performance on \bsuiteversion, obtained by running the experiments from \bsuitegithub\ \cite{osband2019bsuite}.
\end{adjustwidth}
}
}

%%%%%%%%%%%%%%%%%%%%%%%%%%%%%%%%%%%%%%%%%%%%%%%%%%%%%%%%%%%%%%%%%%%%%%%%%%%%%%%%
% CHANGE THESE LINKS
%
% These are convenience macros that provide links to you bsuite material.
% Remember that paths to images should be given *relative* to the file that
% is %%%%%%%%%%%%%%%%%%%%%%%%%%%%%%%%%%%%%%%%%%%%%%%%%%%%%%%%%%%%%%%%%%%%%%%%%%%%%%%%
%%%%%%%%%%%%%%%%%%%%%%%%%%%% BSUITE REPORT TEMPLATE %%%%%%%%%%%%%%%%%%%%%%%%%%%%
%%%%%%%%%%%%%%%%%%%%%%%%%%%%%%%%%%%%%%%%%%%%%%%%%%%%%%%%%%%%%%%%%%%%%%%%%%%%%%%%
%
% Use this LaTeX code to generate an automatic bsuite appendix for your paper
% submission. First \input{bsuite_preamble} before your \begin{document}.
% You can use the bsuite_preamble to define the location of your plots, plus a
% link to the full bsuite comments.
%
% Then, write a short description of your agents in app:bsuite-agents, and a short
% commentary on your results in app:bsuite-commentary.
%
% For some conference formats (e.g. ICLR) it is important to allow margin change
% \includepackage{geometry}, we do not include this in the preamble by default.
%


\newpage
\onecolumn

% If the package does not allow geometry, do not fail
\ifx\newgeometry\undefined\else
\newgeometry{top=20mm, bottom=20mm, left=20mm, right=20mm}
\fi


%%%%%%%%%%%%%%%%%%%%%%%%%%%%%%%%%%%%%%%%%%%%%%%%%%%%%%%%%%%%%%%%%%%%%%%%%%%%%%%%
% TITLE + ABSTRACT [ADD PAPER TITLE]
%
% Macros are defined in bsuite_preamble.tex, use the \label{} to \ref{} from
% other sections in your paper.
\bsuitetitle{Your Paper Title}
\label{app:bsuite-report}
\bsuiteabstract


%%%%%%%%%%%%%%%%%%%%%%%%%%%%%%%%%%%%%%%%%%%%%%%%%%%%%%%%%%%%%%%%%%%%%%%%%%%%%%%%
% AGENT DEFINITION [EDIT]
%
% Use this section to provide a brief overview of the agents that you run on
% bsuite. Usually this will involve links to full descriptions elsewhere in
% your paper.

\subsection{Agent definition}
\label{app:bsuite-agents}
In this experiment all implementations are taken from \url{github.com/deepmind/bsuite/baselines} with default configurations.
We provide a brief summary of the agents run on \bsuiteversion:
\begin{itemize}[noitemsep, nolistsep]
    \item {\bf random}: selects action uniformly at random each timestep.
    \item {\bf dqn}: Deep Q-networks \cite{mnih2015human}.
    \item {\bf boot\_dqn}: bootstrapped DQN with prior networks \cite{osband2016deep,osband2018rpf}.
    \item {\bf actor\_critic\_rnn}: an actor critic with recurrent neural network \cite{mnih2016asynchronous}.
\end{itemize}



%%%%%%%%%%%%%%%%%%%%%%%%%%%%%%%%%%%%%%%%%%%%%%%%%%%%%%%%%%%%%%%%%%%%%%%%%%%%%%%%
% SUMMARY SCORES [DO NOT EDIT]
\subsection{Summary scores}
\label{app:bsuite-scores}

Each \bsuite\ experiment outputs a summary score in [0,1].
We aggregate these scores by according to key experiment type, according to the standard analysis notebook.
A detailed analysis of each of these experiments may be found in a notebook hosted on Colaboratory: \bsuitecolab.

\ifx\newgeometry\undefined\vspace{-2mm}\else\fi % Squeezing for ICML

\begin{figure}[h!]
\centering
\begin{minipage}[t]{.5\textwidth}
  \centering
  \includegraphics[width=\textwidth,height=60mm,keepaspectratio]{\bsuiteradarplot}
  \captionof{figure}{Radar plot gives a snapshot of agent behaviour.}
  \label{fig:radar}
\end{minipage}%
\begin{minipage}[t]{.5\textwidth}
  \centering
  \includegraphics[width=\textwidth,height=60mm,keepaspectratio]{\bsuitebarplot}
  \captionof{figure}{Summary score for each \bsuite\ experiment.}
  \label{fig:bar}
\end{minipage}
\end{figure}

\ifx\newgeometry\undefined\vspace{-2mm}\else\fi % Squeezing for ICML

%%%%%%%%%%%%%%%%%%%%%%%%%%%%%%%%%%%%%%%%%%%%%%%%%%%%%%%%%%%%%%%%%%%%%%%%%%%%%%%%
% RESULTS COMMENTARY [EDIT]

\subsection{Results commentary}
\label{app:bsuite-commentary}

\begin{itemize}[noitemsep, nolistsep, leftmargin=*]
    \item {\bf random} performs poorly across all aspects.
    This confirms that our scoring functions are working as intended.
    \item {\bf dqn} performs well on basic tasks, and quite well on credit assignment, generalization, noise and scale.
    DQN performs extremely poorly across memory and exploration tasks.
    The feedforward MLP has no mechanism for memory, and $\epsilon$=5\%-greedy action selection is notoriously inefficient in domains that require efficient exploration.
    \item {\bf boot\_dqn} performs mostly identically to DQN, except for exploration where it greatly outperforms, and also a smaller boost to performance under noise.
    This result matches our understanding of Bootstrapped DQN as a variant of DQN designed to estimate uncertainty and use this to guide deep exploration.
    \item {\bf actor\_critic\_rnn} typically performs worse than either DQN or Bootstrapped DQN on all tasks apart from memory.
    This agent is the only one able to perform better than random due to its recurrent network architecture.
\end{itemize}


\newpage
. In some cases this will be easier to debug
% if you just copy/paste the tex into your paper file.

\newcommand{\bsuitecolab}{\href{YOUR-LINK-HERE}}  % full bsuite colab report
\newcommand{\bsuiteradarplot}{images/radar_plot}  % path to radar plot
\newcommand{\bsuitebarplot}{images/bar_plot}  % path to bar plot


 is essentially equivalent to
% copy/paste... and in some cases that approach will be much easier to debug!

\usepackage{caption}
\usepackage{changepage}
\usepackage{enumitem}
\usepackage{graphicx}


%%%%%%%%%%%%%%%%%%%%%%%%%%%%%%%%%%%%%%%%%%%%%%%%%%%%%%%%%%%%%%%%%%%%%%%%%%%%%%%%
% DO NOT CHANGE THESE LINKS
%
% These are useful commands that are used in the bsuite_appendix.
% You should not change these from their default values.

\newcommand{\bsuite}{\texttt{bsuite}}
\newcommand{\bsuiteversion}{\texttt{bsuite2019}}
\newcommand{\bsuitegithub}{\url{github.com/deepmind/bsuite}}

% This is a Macro that returns a title = bsuite report: Your Paper Title.
\newcommand{\bsuitetitle}[1]{
\rule{\linewidth}{4pt}
\vspace{-5mm}

\section{\hfil \LARGE \normalfont
\bsuite\ report: #1
\vspace{2mm} \hfil
\vspace{-3mm}
}

\rule{\linewidth}{1pt}
}

\newcommand{\bsuiteabstract}{
{\small
\begin{adjustwidth}{1.5cm}{1.5cm}
The \textit{Behaviour Suite for Core Reinforcement Learning}, or \bsuite\ for short, is a collection of carefully-designed experiments that investigate core capabilities of a reinforcement learning (RL) agent.
The aim of the \bsuite\ project is to collect clear, informative and scalable problems that capture key issues in the design of efficient and general learning algorithms and study agent behaviour through their performance on these shared benchmarks.
This report provides a snapshot of agent performance on \bsuiteversion, obtained by running the experiments from \bsuitegithub\ \cite{osband2019bsuite}.
\end{adjustwidth}
}
}

%%%%%%%%%%%%%%%%%%%%%%%%%%%%%%%%%%%%%%%%%%%%%%%%%%%%%%%%%%%%%%%%%%%%%%%%%%%%%%%%
% CHANGE THESE LINKS
%
% These are convenience macros that provide links to you bsuite material.
% Remember that paths to images should be given *relative* to the file that
% is %%%%%%%%%%%%%%%%%%%%%%%%%%%%%%%%%%%%%%%%%%%%%%%%%%%%%%%%%%%%%%%%%%%%%%%%%%%%%%%%
%%%%%%%%%%%%%%%%%%%%%%%%%%%% BSUITE REPORT TEMPLATE %%%%%%%%%%%%%%%%%%%%%%%%%%%%
%%%%%%%%%%%%%%%%%%%%%%%%%%%%%%%%%%%%%%%%%%%%%%%%%%%%%%%%%%%%%%%%%%%%%%%%%%%%%%%%
%
% Use this LaTeX code to generate an automatic bsuite appendix for your paper
% submission. First %%%%%%%%%%%%%%%%%%%%%%%%%%%%%%%%%%%%%%%%%%%%%%%%%%%%%%%%%%%%%%%%%%%%%%%%%%%%%%%%
%%%%%%%%%%%%%%%%%%%%%%%%%%%% BSUITE LATEX PREAMBLE %%%%%%%%%%%%%%%%%%%%%%%%%%%%%
%%%%%%%%%%%%%%%%%%%%%%%%%%%%%%%%%%%%%%%%%%%%%%%%%%%%%%%%%%%%%%%%%%%%%%%%%%%%%%%%
%
% Use this LaTeX code to generate an automatic bsuite appendix for your paper
% submission. First\input{bsuite_preamble} before your \begin{document}, you can
% fill in the custom \buitecolab, \bsuiteradarplot, \bsuitebarplot to link to
% the necessary bsuite assets for publication.
%
% Next, fill in the necessary sections of bsuite_appendix, and either copy/paste
% or \input{} this into your conference file to create an appendix page.
%
% For some conference formats (e.g. ICLR) it is important to allow margin change
% \includepackage{geometry}, we do not include this in the preamble by default.
%
% Remember that \input{bsuite_preamble.tex} is essentially equivalent to
% copy/paste... and in some cases that approach will be much easier to debug!

\usepackage{caption}
\usepackage{changepage}
\usepackage{enumitem}
\usepackage{graphicx}


%%%%%%%%%%%%%%%%%%%%%%%%%%%%%%%%%%%%%%%%%%%%%%%%%%%%%%%%%%%%%%%%%%%%%%%%%%%%%%%%
% DO NOT CHANGE THESE LINKS
%
% These are useful commands that are used in the bsuite_appendix.
% You should not change these from their default values.

\newcommand{\bsuite}{\texttt{bsuite}}
\newcommand{\bsuiteversion}{\texttt{bsuite2019}}
\newcommand{\bsuitegithub}{\url{github.com/deepmind/bsuite}}

% This is a Macro that returns a title = bsuite report: Your Paper Title.
\newcommand{\bsuitetitle}[1]{
\rule{\linewidth}{4pt}
\vspace{-5mm}

\section{\hfil \LARGE \normalfont
\bsuite\ report: #1
\vspace{2mm} \hfil
\vspace{-3mm}
}

\rule{\linewidth}{1pt}
}

\newcommand{\bsuiteabstract}{
{\small
\begin{adjustwidth}{1.5cm}{1.5cm}
The \textit{Behaviour Suite for Core Reinforcement Learning}, or \bsuite\ for short, is a collection of carefully-designed experiments that investigate core capabilities of a reinforcement learning (RL) agent.
The aim of the \bsuite\ project is to collect clear, informative and scalable problems that capture key issues in the design of efficient and general learning algorithms and study agent behaviour through their performance on these shared benchmarks.
This report provides a snapshot of agent performance on \bsuiteversion, obtained by running the experiments from \bsuitegithub\ \cite{osband2019bsuite}.
\end{adjustwidth}
}
}

%%%%%%%%%%%%%%%%%%%%%%%%%%%%%%%%%%%%%%%%%%%%%%%%%%%%%%%%%%%%%%%%%%%%%%%%%%%%%%%%
% CHANGE THESE LINKS
%
% These are convenience macros that provide links to you bsuite material.
% Remember that paths to images should be given *relative* to the file that
% is \input{bsuite_appendix}. In some cases this will be easier to debug
% if you just copy/paste the tex into your paper file.

\newcommand{\bsuitecolab}{\href{YOUR-LINK-HERE}}  % full bsuite colab report
\newcommand{\bsuiteradarplot}{images/radar_plot}  % path to radar plot
\newcommand{\bsuitebarplot}{images/bar_plot}  % path to bar plot


 before your \begin{document}.
% You can use the bsuite_preamble to define the location of your plots, plus a
% link to the full bsuite comments.
%
% Then, write a short description of your agents in app:bsuite-agents, and a short
% commentary on your results in app:bsuite-commentary.
%
% For some conference formats (e.g. ICLR) it is important to allow margin change
% \includepackage{geometry}, we do not include this in the preamble by default.
%


\newpage
\onecolumn

% If the package does not allow geometry, do not fail
\ifx\newgeometry\undefined\else
\newgeometry{top=20mm, bottom=20mm, left=20mm, right=20mm}
\fi


%%%%%%%%%%%%%%%%%%%%%%%%%%%%%%%%%%%%%%%%%%%%%%%%%%%%%%%%%%%%%%%%%%%%%%%%%%%%%%%%
% TITLE + ABSTRACT [ADD PAPER TITLE]
%
% Macros are defined in bsuite_preamble.tex, use the \label{} to \ref{} from
% other sections in your paper.
\bsuitetitle{Your Paper Title}
\label{app:bsuite-report}
\bsuiteabstract


%%%%%%%%%%%%%%%%%%%%%%%%%%%%%%%%%%%%%%%%%%%%%%%%%%%%%%%%%%%%%%%%%%%%%%%%%%%%%%%%
% AGENT DEFINITION [EDIT]
%
% Use this section to provide a brief overview of the agents that you run on
% bsuite. Usually this will involve links to full descriptions elsewhere in
% your paper.

\subsection{Agent definition}
\label{app:bsuite-agents}
In this experiment all implementations are taken from \url{github.com/deepmind/bsuite/baselines} with default configurations.
We provide a brief summary of the agents run on \bsuiteversion:
\begin{itemize}[noitemsep, nolistsep]
    \item {\bf random}: selects action uniformly at random each timestep.
    \item {\bf dqn}: Deep Q-networks \cite{mnih2015human}.
    \item {\bf boot\_dqn}: bootstrapped DQN with prior networks \cite{osband2016deep,osband2018rpf}.
    \item {\bf actor\_critic\_rnn}: an actor critic with recurrent neural network \cite{mnih2016asynchronous}.
\end{itemize}



%%%%%%%%%%%%%%%%%%%%%%%%%%%%%%%%%%%%%%%%%%%%%%%%%%%%%%%%%%%%%%%%%%%%%%%%%%%%%%%%
% SUMMARY SCORES [DO NOT EDIT]
\subsection{Summary scores}
\label{app:bsuite-scores}

Each \bsuite\ experiment outputs a summary score in [0,1].
We aggregate these scores by according to key experiment type, according to the standard analysis notebook.
A detailed analysis of each of these experiments may be found in a notebook hosted on Colaboratory: \bsuitecolab.

\ifx\newgeometry\undefined\vspace{-2mm}\else\fi % Squeezing for ICML

\begin{figure}[h!]
\centering
\begin{minipage}[t]{.5\textwidth}
  \centering
  \includegraphics[width=\textwidth,height=60mm,keepaspectratio]{\bsuiteradarplot}
  \captionof{figure}{Radar plot gives a snapshot of agent behaviour.}
  \label{fig:radar}
\end{minipage}%
\begin{minipage}[t]{.5\textwidth}
  \centering
  \includegraphics[width=\textwidth,height=60mm,keepaspectratio]{\bsuitebarplot}
  \captionof{figure}{Summary score for each \bsuite\ experiment.}
  \label{fig:bar}
\end{minipage}
\end{figure}

\ifx\newgeometry\undefined\vspace{-2mm}\else\fi % Squeezing for ICML

%%%%%%%%%%%%%%%%%%%%%%%%%%%%%%%%%%%%%%%%%%%%%%%%%%%%%%%%%%%%%%%%%%%%%%%%%%%%%%%%
% RESULTS COMMENTARY [EDIT]

\subsection{Results commentary}
\label{app:bsuite-commentary}

\begin{itemize}[noitemsep, nolistsep, leftmargin=*]
    \item {\bf random} performs poorly across all aspects.
    This confirms that our scoring functions are working as intended.
    \item {\bf dqn} performs well on basic tasks, and quite well on credit assignment, generalization, noise and scale.
    DQN performs extremely poorly across memory and exploration tasks.
    The feedforward MLP has no mechanism for memory, and $\epsilon$=5\%-greedy action selection is notoriously inefficient in domains that require efficient exploration.
    \item {\bf boot\_dqn} performs mostly identically to DQN, except for exploration where it greatly outperforms, and also a smaller boost to performance under noise.
    This result matches our understanding of Bootstrapped DQN as a variant of DQN designed to estimate uncertainty and use this to guide deep exploration.
    \item {\bf actor\_critic\_rnn} typically performs worse than either DQN or Bootstrapped DQN on all tasks apart from memory.
    This agent is the only one able to perform better than random due to its recurrent network architecture.
\end{itemize}


\newpage
. In some cases this will be easier to debug
% if you just copy/paste the tex into your paper file.

\newcommand{\bsuitecolab}{\href{YOUR-LINK-HERE}}  % full bsuite colab report
\newcommand{\bsuiteradarplot}{images/radar_plot}  % path to radar plot
\newcommand{\bsuitebarplot}{images/bar_plot}  % path to bar plot


 before your \begin{document}.
% You can use the bsuite_preamble to define the location of your plots, plus a
% link to the full bsuite comments.
%
% Then, write a short description of your agents in app:bsuite-agents, and a short
% commentary on your results in app:bsuite-commentary.
%
% For some conference formats (e.g. ICLR) it is important to allow margin change
% \includepackage{geometry}, we do not include this in the preamble by default.
%


\newpage
\onecolumn

% If the package does not allow geometry, do not fail
\ifx\newgeometry\undefined\else
\newgeometry{top=20mm, bottom=20mm, left=20mm, right=20mm}
\fi


%%%%%%%%%%%%%%%%%%%%%%%%%%%%%%%%%%%%%%%%%%%%%%%%%%%%%%%%%%%%%%%%%%%%%%%%%%%%%%%%
% TITLE + ABSTRACT [ADD PAPER TITLE]
%
% Macros are defined in bsuite_preamble.tex, use the \label{} to \ref{} from
% other sections in your paper.
\bsuitetitle{Your Paper Title}
\label{app:bsuite-report}
\bsuiteabstract


%%%%%%%%%%%%%%%%%%%%%%%%%%%%%%%%%%%%%%%%%%%%%%%%%%%%%%%%%%%%%%%%%%%%%%%%%%%%%%%%
% AGENT DEFINITION [EDIT]
%
% Use this section to provide a brief overview of the agents that you run on
% bsuite. Usually this will involve links to full descriptions elsewhere in
% your paper.

\subsection{Agent definition}
\label{app:bsuite-agents}
In this experiment all implementations are taken from \url{github.com/deepmind/bsuite/baselines} with default configurations.
We provide a brief summary of the agents run on \bsuiteversion:
\begin{itemize}[noitemsep, nolistsep]
    \item {\bf random}: selects action uniformly at random each timestep.
    \item {\bf dqn}: Deep Q-networks \cite{mnih2015human}.
    \item {\bf boot\_dqn}: bootstrapped DQN with prior networks \cite{osband2016deep,osband2018rpf}.
    \item {\bf actor\_critic\_rnn}: an actor critic with recurrent neural network \cite{mnih2016asynchronous}.
\end{itemize}



%%%%%%%%%%%%%%%%%%%%%%%%%%%%%%%%%%%%%%%%%%%%%%%%%%%%%%%%%%%%%%%%%%%%%%%%%%%%%%%%
% SUMMARY SCORES [DO NOT EDIT]
\subsection{Summary scores}
\label{app:bsuite-scores}

Each \bsuite\ experiment outputs a summary score in [0,1].
We aggregate these scores by according to key experiment type, according to the standard analysis notebook.
A detailed analysis of each of these experiments may be found in a notebook hosted on Colaboratory: \bsuitecolab.

\ifx\newgeometry\undefined\vspace{-2mm}\else\fi % Squeezing for ICML

\begin{figure}[h!]
\centering
\begin{minipage}[t]{.5\textwidth}
  \centering
  \includegraphics[width=\textwidth,height=60mm,keepaspectratio]{\bsuiteradarplot}
  \captionof{figure}{Radar plot gives a snapshot of agent behaviour.}
  \label{fig:radar}
\end{minipage}%
\begin{minipage}[t]{.5\textwidth}
  \centering
  \includegraphics[width=\textwidth,height=60mm,keepaspectratio]{\bsuitebarplot}
  \captionof{figure}{Summary score for each \bsuite\ experiment.}
  \label{fig:bar}
\end{minipage}
\end{figure}

\ifx\newgeometry\undefined\vspace{-2mm}\else\fi % Squeezing for ICML

%%%%%%%%%%%%%%%%%%%%%%%%%%%%%%%%%%%%%%%%%%%%%%%%%%%%%%%%%%%%%%%%%%%%%%%%%%%%%%%%
% RESULTS COMMENTARY [EDIT]

\subsection{Results commentary}
\label{app:bsuite-commentary}

\begin{itemize}[noitemsep, nolistsep, leftmargin=*]
    \item {\bf random} performs poorly across all aspects.
    This confirms that our scoring functions are working as intended.
    \item {\bf dqn} performs well on basic tasks, and quite well on credit assignment, generalization, noise and scale.
    DQN performs extremely poorly across memory and exploration tasks.
    The feedforward MLP has no mechanism for memory, and $\epsilon$=5\%-greedy action selection is notoriously inefficient in domains that require efficient exploration.
    \item {\bf boot\_dqn} performs mostly identically to DQN, except for exploration where it greatly outperforms, and also a smaller boost to performance under noise.
    This result matches our understanding of Bootstrapped DQN as a variant of DQN designed to estimate uncertainty and use this to guide deep exploration.
    \item {\bf actor\_critic\_rnn} typically performs worse than either DQN or Bootstrapped DQN on all tasks apart from memory.
    This agent is the only one able to perform better than random due to its recurrent network architecture.
\end{itemize}


\newpage
